%!TeX root = ../main.tex

\section{Data}
\subsection{Dataset}
The data of the experiment are collected in 10 files for both regimes (still and spinning wheel). Each file contains a dataset with two columns, being the first the time tags in \emph{machine time} ($\simeq 81ps$) and the other the channel, like this:\\
\begin{center}
\begin{tabular}{c c}
0, & 1 \\
18009, & 1 \\
18835, & 1 \\
...  & ...
\end{tabular}
\end{center}
In this case there is only one channel collecting the data so the last column isn't useful. To convert the time tags from machine time to actual time it's sufficient to multiply the first column by $81ps$ obtaining the time tags in seconds. The data can be merged together by putting in column all the datasets and making sure that at each iteration the time begins where it left (for example, if the first dataset ends at $2s$ the second must begin at $2s$).