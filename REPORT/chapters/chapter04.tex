%!TeX root = ../main.tex

\section{Momenta Analysis}
A statistical comparison between the theoretical and empirical model can be performed by analyzing superior orders of the distributions momenta.
\subsection{Poisson distribution}
The first four momenta are described as:
\begin{align*}
m_{1,P}=\sum_{n} \dfrac {n\cdot e^{-\lambda}\cdot \lambda^n}{n!} &\implies \lambda\\
m_{2,P}=\sum_{n} \dfrac {(n-\lambda)^2\cdot e^{-\lambda}\cdot \lambda^n}{n!} &\implies \lambda \\
m_{3,P}=\sum_{n} \dfrac {(n-\lambda)^3\cdot e^{-\lambda}\cdot \lambda^n}{n!} &\implies \lambda \\
m_{4,P}=\sum_{n} \dfrac {(n-\lambda)^4\cdot e^{-\lambda}\cdot \lambda^n}{n!} &\implies \lambda+3\lambda^2 
\end{align*}


where $\lambda$ is the mean of the Poisson distribution and $n=0,1,2..$ . The comparison can be done by subtracting the terms on the left with the terms on the right. The results are:
\begin{align*}
|m_{1,P}-\lambda|& \simeq 1.0\cdot10^{-5} \\
|m_{2,P}-\lambda|& \simeq 1.4\cdot10^{-4} \\
|m_{3,P}-\lambda|& \simeq 2.6\cdot10^{-3} \\
|m_{4,P}-(\lambda+3\lambda^2 )|& \simeq 4.5\cdot10^{-2}
\end{align*}


Values can be affected by the choice of $n$: the more $n$ increases, the more the differences become close to zero (for the calculations was set to its maximum value).

\subsection{Bose-Einstein distribution}
The first four momenta are described as:
\begin{align*}
m_{1,BE}=\sum_n\dfrac{n}{1+\lambda}\cdot\left(\dfrac{\lambda}{1+\lambda}\right)^n &\implies \lambda \\
m_{2,BE}=\sum_n\dfrac{(n-\lambda)^2}{1+\lambda}\cdot\left(\dfrac{\lambda}{1+\lambda}\right)^n &\implies \lambda+\lambda^2 \\
m_{3,BE}=\sum_n\dfrac{(n-\lambda)^3}{1+\lambda}\cdot\left(\dfrac{\lambda}{1+\lambda}\right)^n &\implies \lambda+3\lambda^2+2\lambda^3\\
m_{4,BE}=\sum_n\dfrac{(n-\lambda)^4}{1+\lambda}\cdot\left(\dfrac{\lambda}{1+\lambda}\right)^n &\implies \lambda+10\lambda^2+18\lambda^3+9\lambda^4\\
\end{align*}

where $\lambda$ is the mean of the Bose-Einstein distribution and $n=0,1,2..$ . The comparison can be done by subtracting the terms on the left with the terms on the right. The results are:
\begin{align*}
|m_{1,BE}-\lambda| &\simeq 1.3\cdot10^{-9} \\
|m_{2,BE}-(\lambda+\lambda^2)| &\simeq 9.0\cdot10^{-8} \\
|m_{3,BE}-(\lambda+3\lambda^2+2\lambda^3)| &\simeq 6.3\cdot10^{-6} \\
|m_{4,BE}-(\lambda+10\lambda^2+18\lambda^3+9\lambda^4)| &\simeq 4.5\cdot10^{-4}
\end{align*}

Values can be affected by the choice of $n$: the more $n$ increases, the more the differences become close to zero (for the calculations was set to its maximum value).