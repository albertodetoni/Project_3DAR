%!TeX root = ../main.tex

\section{Conclusions}
The plots and the momenta analysis display a strong similarity between the data collected and the theoretical models. The Arecchi experiment exhibits the difference between a coherent and a thermal source: the statistical description of a coherent source follows a Poisson distribution of mean $\lambda$, while a thermal source follows a Bose-Einstein distribution of parameter $\lambda$. This can be verified empirically because by reducing the time bin the less photons are collected and the Bose-Einstein distribution follows this kind of behaviour (strong peak around $0$), while the Poisson distribution tends to keep a bell-shape around the mean (like a coherent light source) and by changing $T$ all it does is shifting and spreading the ``bell''. Note that the differences between the empirical and theoretical momenta are really small: the twos can be considered approximately equal despite some error due to the small set of $n$.